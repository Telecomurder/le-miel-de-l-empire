\documentclass[14pt,twocolumn]{extarticle}
\usepackage[utf8]{inputenc}
\usepackage[T1]{fontenc}
\usepackage[francais]{babel}

\setlength{\columnsep}{0.5in}

\usepackage{geometry}
\geometry{a4paper}
\geometry{margin=1in}

\usepackage{times}

\usepackage{titlesec}
\titleformat{\section}[frame]
{\normalfont} {} {15pt} {\itshape\bfseries\filcenter}
\titleformat{\subsection}
{\normalfont} {} {1pt} {\itshape\bfseries}

\pagestyle{empty}

\begin{document}

\section{DUC LIADENN DE ROMAINE\\(DUCHESSE LIADENN DE ROMANE)}

\subsection{LE PERSONNAGE :}

Bien que vous ne soyez pas le coupable, la mort de l'empereur Hélios~III
arrange bien vos affaires.

Pendant que les dignitaires de l'Empire s'entre-déchireront sur les problèmes
de succession à Malte et dans le reste de l'Empire, ils ne viendront pas
fouiner dans les affaires de la famille Romane. Cette affaire vous soulage car
vous n'êtes pas le vrai duc Liadenn de Romane mais son cousin.

L'histoire est à la fois simple et compliquée. Le côté simple est que votre
cousin, héritier légitime du titre de duc de Liadenn, est décédé alors qu'il
n'était encore qu'un bébé et qu'il était le seul de la famille Romane à pouvoir
prendre le titre à la mort de son père. Sa mort aurait dû avoir pour
conséquence le changement de lignée à une autre banche de la famille pour le
titre de Duc.

Le côté compliqué de l'affaire est que le conseil de la famille Romane a
préféré trouver une autre solution pour garder le pouvoir et cette solution a
consisté à vous <<~échanger~>> avec votre cousin. Bien évidemment, cela a été
réalisé dans la plus totale discrétion au prix de quelques morts
<<~accidentelles~>>, mais l'Empire dispose de toutes les caractéristiques
génétiques des nobles de l'Empire et si l'envie prenait à l'empereur de
comparer vos empreintes génétiques à celles de l'ancien duc de Liadenn, il
s'apercevrait vite que non seulement vous n'êtes pas son fils mais que vous
n'êtes pas non plus le fils de l'ancienne duchesse. Pour être plus clair vous
seriez déclaré usurpateur et passible de la peine de mort.

Le but de votre famille est de garder le pouvoir et de le transmettre à vos
héritiers. En effet, une fois que vous aurez transmis le titre à votre fils, il
sera très difficile de découvrir la vérité.

En attendant, vous devez faire en sorte de ne pas vous attirer l'inimitié de
l'empereur ou même faire trop concurrence à un autre duc afin qu'ils ne
recherchent pas dans votre passé quelques détails compromettants.

Mais vous devez aussi sauvegarder les intérêts du duché et bien sûr être en
conflit avec vos nobles voisins voire avec l'Empire. Autant dire que vous ne
prenez aucune décision sans l'avoir mûrement réfléchie avec ou sans l'appui du
conseil de famille.

Toutes ces tensions vous ont depuis longtemps amené à penser qu'une guerre de
succession au sein de l'empire serait le meilleur moyen d’empêcher que l'on
fouille votre passé. Même si la vérité sur votre naissance était découverte,
vous pourriez faire croire qu'il s'agit d'une vile man\oe{}uvre de vos
adversaires pour diviser le duché. Vous aviez donc un excellent mobile
d'assassiner l'Empereur et vous avez préparé le duché à un éventuel conflit.

Malgré cette préparation militaire, vous avez plutôt \oe{}uvrer pour le
développement économique du duché en vue de l'amélioration du niveau de vie de
vos sujets. D'autre part, vous avez essayé de convaincre l'Empereur de faire
changer les clauses de passation du titre ducal, et vous vous êtes disputé
lorsque qu'il a refusé.

\subsection{VOS OBJECTIFS :}

Vous rechercherez au cours de la soirée le coupable du meurtre et ses
motivations.

Vous devrez garder secret vos origines réelles.

Vous devrez semer la zizanie chez les autres personnages et faire en sorte
qu'il n'y ait pas de nomination de duc de Malte ou le nec plus ultra est que ce
soit vous qui obteniez le titre.

Il serait bien que vous arriviez à faire signer cette charte de succession
ducale afin d'assurer vos arrières. Vos objectifs plus personnels sont de
déconsidérer Deleb Karnar parce que vous vous estimez trahi par sa démission et
de vous opposer au duc de Rodrigo parce son style est l'opposé du votre, qu'il
vous a débauché votre maître de guerre (Deleb Karnar) et que vous détestez sa
façon de gouverner son duché.

\subsection{L'INTERPRÉTATION DU PERSONNAGE :}

Vous jouez sur une corde raide entre deux objectifs contradictoires : rester
discret pour ne pas avoir d'ennemi et défendre le duché contre les intérêts des
nobles et de l'empire. Nous vous conseillons de vous faire accepter par toutes
les parties sans vous engager réellement à quoi que ce soit.

Vous devrez veiller à ce que votre secret le reste et pour cela votre stratégie
consistera à opposer les personnages entre eux en attisant les haines ou
antipathies naturelles. Vous méprisez votre ancien maître de guerre et vous
aurez du mal à le dissimuler.

N'oubliez pas cependant que vous êtes un duc et l'un des personnages les plus
importants de l'empire. Vous vous devez d'assurer la continuité de la
succession dans la famille Romane.

Cet objectif est compris dans ceux qui précèdent.

Le plus important est de vous amuser et nous vous laissons le soin de trouver
la tenue que vous jugerez la plus adéquate pour vous sentir bien dans la peau
de votre personnage (déguisement de SF, tenue BCBG, gentleman-farmer, tenue de
soirée...).

\subsection{VOS TALENTS PARTICULIERS:}

Vous avez dirigé brillamment les affaires de votre duché, mais ceci ne
constitue pas un talent particulier.

Vous pouvez fouiller les pièces de la station moyennant 1~PA par fouille. Vous
ne pouvez pas fouiller vos appartements car vous êtes censé connaître ce qui
s'y trouve. Ce qu'il y a de compromettant : une photo de votre mère, une note
interne sur un déploiement de troupes et recrutement en vue d’un conflit
important, et une proposition à l’Empereur visant à transformer le titre ducal
héréditaire en test plus général sur toute la famille ducale.

\subsection{ACCESSOIRES :}

Vous possédez une dose de Miel de Malte, fictivement dissimulée dans votre
pièce et gérée par l'organisateur. Vous devrez donc l'avertir de toute
utilisation de cette dose.

\end{document}

