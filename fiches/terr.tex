\documentclass[14pt,twocolumn]{extarticle}
\usepackage[utf8]{inputenc}
\usepackage[T1]{fontenc}
\usepackage[francais]{babel}

\setlength{\columnsep}{0.5in}

\usepackage{geometry}
\geometry{a4paper}
\geometry{margin=1in}

\usepackage{times}

\usepackage{titlesec}
\titleformat{\section}[frame]
{\normalfont} {} {15pt} {\itshape\bfseries\filcenter}
\titleformat{\subsection}
{\normalfont} {} {1pt} {\itshape\bfseries}

\pagestyle{empty}

\begin{document}

\section{ARGANN TERR\\(ARGANNE TERR)}

\subsection{LE PERSONNAGE~:}

Soyons francs~: vous n'êtes pas coupable de régicide, mais la mort de
l'Empereur ne vous émeut pas du tout, et elle vous permet d'envisager l'avenir
sous de meilleures auspices. En effet, vous avez été le cerveau humain le plus
brillant durant le règne d'Hélios, et vos avis étaient le plus souvent suivis
par l'Empereur et ses ministres, jusqu'à ces dernières années, où l'Empereur se
défiait de certaines de vos déductions, qui entrainaient des prises de
positions trop radicales.

Vous savez que vos rapports étaient orientés en fonction de vos sentiments, et
non pas des faits et des logiques induites par votre raisonnement exceptionnel.
En fait, vous êtes devenu dépendant du Miel comme d'une drogue, car, dans votre
ambition de rester au sommet de l'Empire, vous avez sacrifié votre corps
physique sur l'autel de la Connaissance.

Malheureusement, Hélios~III s'en est aperçu, et vous saviez que cette mission
sur Carbon~B serait la dernière avant votre mise à la retraite forcée.

Bien qu'Hélios soit mort, vous ne pouvez compter sur son fils, Andrev, pour
garder votre place, car un ensemble de faits semble vous prouver qu'il fera
appel à un jeune et brillant intellect pour être au service de l'Empire et de
l'Empereur.

Parce qu'il ne vous estime pas à votre juste valeur et qu'il est prétentieux et
irritant, vous haïssez Andrev, et ne perdrez aucune occasion de lui prouver à
quel point il a eu tort de ne pas vous faire confiance.

Mais cette inimitié n'est rien comparé à l'aversion profonde, constante et
affichée que vous éprouvez envers tout ce qui à trait aux Pythonisses. Vous
partagez pleinement le point de vue de l'ensemble de vos collègues, qui
méprisent le prétendu pouvoir des Pythonisses, basé, au mieux sur un premier
niveau d'analyse comportementale, et beaucoup sur les analyses que vous ou vos
collègues avaient faites. Qu'elle soit la première Pythonisse ne rend les
confrontations avec Dame Perduinn que moins déséquilibrées, mais elles tournent
toujours en votre faveur, même si l'Empereur ou son fils ont toujours tempéré
vos propos, et parfois agi dans le sens préconisé par la Pythonisse.

Mais vous savez au fond que ce n'est qu'une attitude politique et pas toujours
consciente de la part de personnes, certes intelligentes, mais largement loin
d'atteindre vos capacités lorsque vous êtes en <<~réflexion~>> avec l'aide du
Miel.

\subsection{VOS OBJECTIFS:}

Avant toute chose, il vous faut trouver un poste auprès d'un puissant. Et
puisque vous ne pourrez plus, d'après vos déductions, exercer vos talents
auprès du nouvel Empereur, vous ne retrouverez plus de si tôt l'occasion de
discuter physiquement et aussi longuement que maintenant avec les puissants de
l'Empire.

Vous devez être assez convaincant pour que l'un des personnages présents vous
signe un contrat de travail.

Tout contrat signé avec un membre de votre congrégation stipule forcément
l'octroi d'une dose mensuelle de Miel de Malte, mais vous devez en obtenir,
pour survivre, au moins deux par mois, et trois dans un proche avenir.

À vous de négocier fermement, car vous savez aussi qu'un tel poste peut offrir
certains avantages et une protection certaine vis-à-vis des autorités.

Vos autres objectifs, aussi importants, bien qu'ils ne soient pas d'ordre
alimentaire, est de faire rendre gorge à Andrev de sa décision inexprimée et
imbécile de vous virer comme un laquais sans importance, et de rabaisser à son
rôle de liseuse d'avenir dans une boule de cristal opaque cette Dame Perduinn,
qui est un affront permanent à la logique et au bon sens.

\subsection{L'INTERPRÉTATION DU PERSONNAGE~:}

Vous êtes drogué, et vous devez montrer un comportement nerveux et fébrile,
mais, en tant qu'intellect, vous avez un coté savant fou très marqué.

Préparez quelques répliques fumeuses sur quelques sujets, et replacez les sans
à propos dans vos conversations. Par exemple~: <<~L'utilisation de
l'espace-temps est un facteur de développement certain de l'Empire, par
l'augmentation des capacités de compréhension des facteurs déterminants.~>>

Vous devez prouver votre valeur à vos futurs employeurs, mais rien ne vous
interdit d'en profiter pour leur demander une contribution sous forme de dose
de Miel.

Vous devez avoir une attitude de savant fou extrêmement précieux par la qualité
de ses réflexions lorsque celles-ci correspondent bien aux préoccupations du ou
des interlocuteurs présents. À vous d'utiliser vos découvertes faites au cours
de la soirée et de les mettre en valeur.

N'oubliez pas de rabaisser la Pythonisse à chaque occasion, et de rappeler à
Andrev que votre connaissance de l'Empire est supérieure à la sienne. Quant à
son expérience, la sagesse voudrait qu'il s'appuie sur vous pour son début de
règne.

\subsection{VOS TALENTS PARTICULIERS~:}

Vous pouvez, moyennant 3~PA, effectuer une réflexion profonde d'intellect,
c'est-à-dire répondre par oui ou par non à une question.

Cette opération se déroule en présence du seul organisateur. Vous pouvez
ensuite mentir aux autres personnages qui vous aurait commandité en vous
payant en tout ou partie le cout de la réflexion en PA, en Miel ou d'une autre
façon.

Vous pouvez fouiller une pièce en contrepartie d'1~PA, à l'exception de vos
appartements que vous connaissez.

\end{document}
