\documentclass[14pt,twocolumn]{extarticle}
\usepackage[utf8]{inputenc}
\usepackage[T1]{fontenc}
\usepackage[francais]{babel}

\setlength{\columnsep}{0.5in}

\usepackage{geometry}
\geometry{a4paper}
\geometry{margin=1in}

\usepackage{times}

\usepackage{titlesec}
\titleformat{\section}[frame]
{\normalfont} {} {15pt} {\itshape\bfseries\filcenter}
\titleformat{\subsection}
{\normalfont} {} {1pt} {\itshape\bfseries}
\titleformat{\subsubsection}[frame]
{\normalfont} {} {15pt} {\itshape\bfseries\filcenter}

\pagestyle{empty}

\begin{document}

\section{SYNOPSIS}

\subsection{PRÉAMBULE~:}

La toile de fonds de ce scénario est librement inspiré de \textit{Dune}, le
film de David \textsc{Lynch}, tiré des romans de Frank \textsc{Herbert}.

Pour ceux qui ne connaissent pas, nous décrivons brièvement cette toile de fond 
au scénario. Qu'ils soient ici rassurés, l'essentiel du jeu est dans leur
capacité d'interpréter le personnage et de réaliser les objectifs qui leur
sont fixés.

\subsection{ENVIRONNEMENT~:}

Notre galaxie dans un futur lointain... L'Empire humain règne sur des milliers
de planètes, mais n'a pas rencontré de races intelligentes, et l'empire s'étend
au fur et à mesure de la découverte de systèmes solaires et planétaires offrant
un ensemble de paramètres suffisants pour permettre une colonisation humaine.
Ces colonies exploitent ensuite les ressources des autres planètes proches.

L'ensemble des déplacements galactiques est assuré par la Guilde des Voyageurs,
qui est indépendante de tout système planétaire et politique, y compris de
l'Empereur. L'ensemble des systèmes planétaires est sous l'autorité suprême de
l'Empereur Hélios~III, qui règne sur ses vassaux, dont les plus puissants sont
les Ducs qui dirigent une grande partie de l'Empire~: les duchés. Ce système
féodal est le résultat des multiples guerres qui ont eues lieu après la
première phase de colonisation partie de la Terre. Il permet d'offrir une paix
relative à l'ensemble de la population humaine depuis des siècles.

La loi impériale interdit en effet tout combat sur les planètes colonisées,
ainsi que les attaques sur des vaisseaux non militaires. Les nobles n'utilisent
leur force militaire que pour s'assurer de la conquête d'un système colonisable
ou renfermant des richesses minières.

Les forces militaires sont constituées par les nobles, et obéissent aux
directives d'un Maitre de Guerre, qui gère la protection et l'expansion du
territoire galactique de son seigneur Comte ou Duc. Le Maitre de Guerre
est le titre d'officier le plus élevé, et est toujours confié en fonction des
compétences éprouvées de l'officier. Parmi ces milliers de colonies, une
petite planète fait l'objet de toutes les attentions et convoitises~: Malte.
C'est la seule planète de tout l'Empire où l'on puisse récolter le Miel,
produit miraculeux qui permet de prolonger la vie et accomplir des prodiges.

Le Miel permet à la Guilde des Voyageurs de réduire à quelques jours les
voyages intergalactiques, par une technique mentale qui replie physiquement
l'espace pour les vaisseaux. Cette découverte liée au Miel a permis les
relations entre les différentes colonies, ce qui explique que le Miel est le
bien le plus important, et constitue la monnaie d'échange de référence.

L'autre utilisation du Miel est celle qui accroît ou transforme les capacités
du cerveau humain.

Les intellects s'en servent pour acquérir une puissance de déduction
instantanée, dépassant largement le temps de réponse des ordinateurs biotiques.

La Voie utilisée par les Pythonisses est plus mystérieuse, même si l'on sait
qu'elles utilisent le Miel pour connaître la vérité. D'ailleurs, l'Empereur
Hypérion avait cette sentence lorsque l'on évoquait devant lui le pouvoir des
Pythonisses~: "La vérité n'emprunte pas forcément le plus court chemin de la
parole, alors que le Miel se dissout rapidement dans le sang ; fut-il du
sang de Pythonisse".

Malte dépend de l'autorité d'un Duc, nommé par l'Empereur pour une durée de
50~ans. Cette désignation est destinée à rétablir un équilibre entre les Ducs,
car c'est le plus puissant d'entre eux qui est généralement choisi, et il doit
concentrer ses efforts sur Malte, et donc freiner l'expansion de son Duché.
Bien sûr, les compensations sont à la mesure des sacrifices.

Depuis la création de l'Empire, une seule fois l'Empereur régnant s'est
auto-proclamé régent temporaire de Malte. Il a dû rapidement cédé la gestion de
Malte à un Duc, devant la coalition des nobles qui menaçait de le destituer.

Nous sommes justement à la veille du traditionnel choix du nouveau Duc de Malte
par l'Empereur Hélios~III.

Les personnalités les plus importantes de l'Empire sont réunies sur un petit
satellite artificiel de Malte, l'astroport Carbon~B.

\subsection{JUSTE AVANT LE DÉBUT DE LA SOIRÉE~:}

Peu avant que débute la cérémonie officielle de désignation, la voix de
l'Empereur Hélios~III se fait entendre sur le circuit audio~:

"C'EST UN ASSASSINAT..."

puis plus rien.

Après vous être dirigé vers les appartements de l'Empereur, vous apercevez
celui-ci allongé sur son lit, manifestement mort.

Vous êtes, avec toutes les autres personnalitésn enfermé dans un quartier de
sécurité totale~:le coupable ne peut être que l'un d'entre vous. Mais qui~?
Le découvrir est le premier objectif de tous les personnages (à l'exception du
coupable bien sûr).

\clearpage

\subsection{LES PERSONNALISTÉS~:}

\subsubsection{B2NPC}

Droïde de cérémonie chargé du service lors de la conférence au sommet désignant
le futur Duc de Malte. Du fait de sa programmation et des protections
innombrables de ses circuits vitaux, il est impossible que ce soit lui le
coupable (ce rôle est dévolu à l'organisateur de la soirée).

\subsubsection{PRINCE ANDREV\\(PRINCESSE ANDREA)}

C'est le fils de l'Empereur, héritier du titre, qui a pris depuis peu d'années
ses fonctions auprès de son père.

Sa personnalité est peu connue, car le règne d'Hélios~III devait durer
encore une vingtaine d'année avant qu'il lui transmette son titre lors du
cinquantième anniversaire du prince, comme le veut la coutume.

\subsubsection{DAME PERDUINN, RÉVÉRENDE MÈRE}

Elle représente les Pythonisses, ordre mi-religieux, mi-politique, qui pèse de
tout son poids sur les orientations de l'Empire. Les Pythonisses ont le pouvoir
de lire dans les pensées, ou de connaître la vérité dans certaines conditions.
Même si elles ne sont pas vénérées, elles sont généralement craintes.

La première d'entre elles, Dame Perduinn, est jeune mais l'on dit que son
pouvoir n'en est donc que plus grand, et que l'ordre n'en deviendra que plus
écouté.

\subsubsection{MAITRE KORRIG\\(MAITRESSE KORRIGA)}

C'est le représentant de la Guilde des Voyageurs pour cette réunion. Korrig
est souvent désigné lorsque les intérêts de la Guilde doivent être représentés
parmi les plus hautes personnalités de l'Empire.

La Guilde, organisme indépendant de tout pouvoir, y compris celui de
l'Empereur, gère l'intégralité des déplacements galactiques au sein de
l'Empire.

La Guilde cultive l'art du secret depuis la découverte, par le Premier maitre,
des pouvoirs du Miel. C'est dire combien il est difficile d'en savoir plus que
ce que les maitres veulent bien en révéler.

\subsubsection{DELEB KARNAR\\(HELEN KARNAR)}

C'est le Maitre de Guerre du Duc Igane de Rodrigo. Sa réputation de stratège
est excellente, mais elle est peut-être également due au fait que son Duc règne
sur Malte, ce qui lui permettrait de se voir alloué des moyens supérieurs à
ceux de ses voisins et ennemis.

C'est à peu près les seuls renseignements que vous possédiez sur lui, si ce
n'est qu'il était précédemment au service du Duc Liadenn.

\subsubsection{NORETH CAL NORETH\\(NORESSE CAL NORESSE)}

C'est le Khân des Khâns des Hommes Libres de Malte, c'est-à-dire le chefs de
tous les clans des autochtones. En fait, pour la plupart des personnalités, il
représente la main d'œuvre spécialisée et indispensable à la récolte du Miel.

Mais vous savez qu'il vaut mieux négocier que contraindre, et sa présence lors
de la désignation du nouveau Duc de Malte est purement formelle.

\subsubsection{ARGAN TERR\\(ARGANNE TERR)}

C'est l'intellect de l'Empereur Hélios, et donc son premier conseiller. Du fait
de ses fonctions, il possède un pouvoir important, même si, comme la majeure
partie de ses collègues, son comportement n'est pas toujours très
compréhensible. Pour avoir fréquenter d'autres intellects, vous ne savez pas
vraiment distinguer ce qui relève des effets secondaires de l'absorption
continuelle de Miel de ce qui appartient à l'art de paraître dérangé pour mieux
étudier les situations et les personnes.

\subsubsection{DUC IGANE DE RODRIGO\\(DUCHESSE IGANE DE RODRIGO)}

C'est le Duc régnant sur Malte actuellement, ce qui le situe en deuxième
position, après l'Empereur, dans l'ordre de puissance. Ses méthodes, que l'on
dit brutales, n'ont pas contribué à sa popularité. Mais être Duc de Malte
suscite bien des jalousies, et tout ce qui peut discréditer ce Duc est bon à
prendre, à amplifier et à déformer pour tous les envieux ou ennemis déclarés ou
cachés.

\subsubsection{DUC LIADENN DE ROMANE\\(DUCHESSE LIADENN DE ROMANE)}

C'est l'un des deux prétendants au titre de Duc de Malte, et donc l'un des
personnages les plus importants de l'Empire. On prétend que ce Duc est plus
occupé à développer son Duché qu'à étendre les frontières de l'Empire. Mais il
est évident qu'il fait l'objet, lui aussi, d'une campagne de dénigrement, du
fait de sa puissance.

L'Empereur, qui est vraisemblablement le mieux informé, l'a fait figurer sur la
liste réduite des prétendants.

\subsubsection{DUC MIKA DE VARLEY\\(DUCHESSE MIKA DE VARLEY)}

Le deuxième prétendant au titre de Duc de Malte. Il a hissé son Duché au plus
haut sommet, en accroissant considérablement son territoire, et donc celui de
l'Empire. Ses détracteurs disent que les colons de son Duché n'ont que le
minimum requis pour vivre, et qu'ils payent cher la politique d'expansion du
Duc.

Cela étant dit, aucune planète n'a fait de requête auprès de l'Empereur, et ce
dernier voit dans ce jeune Duché un rival potentiel, puisqu'il l'a pressenti
comme candidat potentiel au titre de Duc de Malte.
\framebox[\columnwidth]{}

\end{document}

