\documentclass{article}
\usepackage[utf8]{inputenc}
\usepackage[T1]{fontenc}
\usepackage[francais]{babel}

\usepackage[top=0cm,bottom=0cm,left=2cm,right=2cm]{geometry}

\title{Aide pour l'organisateur}
\date{}

\pagestyle{empty}

\begin{document}

\maketitle

\section*{Les pouvoirs~:}

\begin{description}
    \item[Divination des Pythonisses] Dame Perduinn oblige un autre personnage
        à lui dire la stricte vérité sur un sujet précis, pendant une minute.
        Attention~: la Pythonisse doit laisser parler son interlocuteur,
        \emph{sans l’interrompre}.  Cela se fait à part devant, l’organisateur
        (il s'agit en fait un contact esprit à esprit, et les autres
        personnages ne <<~voient~>> donc rien).
        
        Coût : 3~PA/divination. La Pythonisse ne peut faire que trois
        divinations dans la soirée. Elle peut mentir aux autres sur la
        divination.

    \item[Pouvoir du Khân des Hommes Libres] Ce pouvoir est équivalent au
        pouvoir de la Pythonisse.

        Coût~: 2~doses de Miel/divination. Le Khân peut se faire payer (en
        Miel) par un autre joueur pour faire une divination.

    \item[Détection du Miel] Le Khân peut également <<~détecter le Miel~>> au
        lieu de se livrer à une fouille de pièce, et décrouvrir une dose, s'il
        y en a une dans la pièce.

        Coût~: 1~PA/détection.

    \item[Réflexion de l'intellect] L’intellect peut, par pure réflexion,
        répondre de façon certaine à une question, par oui ou par non. Cela se
        fait à part devant l’organisateur (c’est en fait une réflexion de
        l’intellect et les autres personnages ne <<~voient~>> donc rien).
        
        Coût : 3~PA/réflexion. L’intellect peut se faire payer par un autre
        joueur pour faire une réflexion. Il peut mentir sur le résultat.

\end{description}

\section*{La possession de Miel~:}

Les doses de Miel sont rangées dans les chambres des personnages qui les
possèdent. Ils ne peuvent pas les garder sur eux dans la soirée. C’est
l’organisateur qui gère les possessions, sur un petit plan du Quartier de
Sécurité Totale. Il est possible pour d'autres personnages que le Khân de
voler des doses~: dans une pièce où se trouve une dose~: un jet de 1D6~: si
le résultat est 1, alors le joueur trouve une dose au lieu d’un indice. Coût~:
1~PA.

\begin{center}
    \begin{tabular}{|c|c|}
        \hline
        Personnage possédant du Miel & Quantité de Miel possédée (en dose)\\
        \hline
        Prince Andrev & 1 \\
        Maitre Korrig & 2 \\
        Deleb Karnar & 1 \\
        Noreth Cal Noreth & 2 \\
        Duc Igane de Rodrigo & 1 \\
        Duc Liadenn de Romane & 1 \\
        Duc Mika de Varley & 1 \\
        \hline
    \end{tabular}
\end{center}

\section*{L'accès à la salle informatique~:}
\begin{center}
    \begin{tabular}{|c|c|}
        \hline
        Personnage ayant accès à la salle informatique & Coût (en PA)\\
        \hline
        Maitre Korrig & 2 \\
        Deleb Karnar & 1 \\
        Duc Mika de Varley & 1 \\
        \hline
    \end{tabular}
\end{center}

\end{document}

