\documentclass{article}
\usepackage[utf8]{inputenc}
\usepackage[T1]{fontenc}
\usepackage[francais]{babel}

\usepackage[top=2cm,bottom=2cm,left=2cm,right=2cm]{geometry}
\usepackage{framed}

\addto\captionsfrancais{\renewcommand{\figurename}{\textsc{Indice salle
    informatique}}}
\usepackage[labelsep=none]{caption}
\usepackage[usenames,dvipsnames]{color}

\pagestyle{empty}

\newcommand{\indice}[3]{
    \begin{figure}[H]
        \begin{center}
            \rule{0.5\textwidth}{1pt}
        \end{center}
        \begin{framed}
            \begin{samepage}
                \texttt{Rapport #1}
                \nopagebreak

                [...]~#2~[...]

                \nopagebreak
                \hfill\textit{#3}
            \end{samepage}
        \end{framed}
        \caption{}
        \begin{center}
            \rule{0.5\textwidth}{1pt}
        \end{center}
    \end{figure}
}

\begin{document}
\indice
{DYJ-52361-OQR/78901}
{
    Les observations répétées sur les droïdes \og Naturel Protocolaire
    type~C\fg{} (droïdes NPC) que j'ai menées avec mon équipe ont révélé un
    phénomène unique et assez troublant: les NPC étudiés ont une tendance
    systématique à la déconnexion dès que les exigences de leur fonction le
    leur permettent. Ce qui est chez tous les autres droïdes un simple soucis
    d'économie de fonctionnement semble chez les NPC avoir tourné à la manie.
    Interrogés par mes soins, les NPC ont nié accomplir ces déconnexions pour
    leur plaisir. Cependant, leur niveau résiduel d'énergie reste élevé. La
    question est donc posée: mais que font les NPC déconnectés? Ces droïdes
    rêveraient-ils de moutons électriques?
}{Professeur P.K. Dick, chargé de recherche à l'université Ubik~4.}

\indice
{LEG-63182-PSG/31458}
{
    L'éducation très stricte et l'éloignement constant de son père ont
    développé chez le jeune Prince Andrev des comportements de repli sur soi et
    de recherche de la solitude. Cette attitude n'a, en soi, rien de
    particulièrement remarquable, mais on note cependant chez le Prince une
    tendance à l'individualisme et au rejet de l'autorité (surtout parentale),
    qui peut amener à s'interroger sur ses capacités à s'intégrer
    harmonieusement au microcosme du Palais.
}{Professeur Lo Rôn, Psychologue en chef du Palais Impérial.}

\indice
{LDR-79135-MEG/08094}
{
    Les circonstances exactes de la mort de la Révérende Mère Drowen
    demeurent et demeureront sans doute toujours mystérieuses. Cependant, la
    thèse officielle de l'accident cardiaque peut être mise en doute.~[...]

    \nobreak

    [...]~Même si l'emploi des pouvoirs des Pythonisses ne peut être démontré,
    les analyses chimiques effectuées sur le cadavre lors de l'autopsie (et
    retirées du dossier d'expertise final) sont troublantes.~[...]

    \nobreak

    [...]~La présence de plusieurs Pythonisses lors du décès de Drowen et leur
    attitude étrange (en particulier celle de Sœur Perduinn) continue de
    laisser planer un doute.
}{Loydd Biggle, Enquêteur Impérial.}

\indice
{QPM-14967-FUX/35748}
{
    Josua Korrig, mort par pendaison le 11/14/1295, exécuté sur ordre du Duc
    Mika de Varley avec plusieurs de ses complices pour trafic de Miel sur la
    planète Gaborii~3 (a toujours nié les faits qui lui étaient reprochés).
}{}

\indice
{FND-79135-OGZ/20530}
{
    Les tests effectués sur simulateurs démontrent clairement l'importance
    cruciale des décisions prises par le commandant en chef qui surpassent par
    exemple le potentiel de feu des bâtiments ou leur nombre.~[...]

    \nobreak

    [...]~Les capacités d'analyse et de commandement du Maitre de Guerre Deleb
    Karnar dépassent très largement les normes habituelles. On peut
    raisonnablement estimer qu'une flotte de combat de taille standard dirigée
    par Karnar serait à même de vaincre tout autre flotte par la seule présence
    à sa tête d'un tacticien de sa dimension.
}{Commodore Klaus Witz, Grand Amiral de la flotte Impériale.}

\indice
{HIB-30176-JEH/61058}
{
    Les velléités d'indépendance des habitants de Malte (les \og Hommes
    Libres\fg{}) ne sont un secret pour personne. Il est donc de notre ressort
    de surveiller attentivement tous les activistes qui pourraient nuire à la
    sureté de l'Empire.~[...]

    \nobreak

    [...]~Parmi les leaders du mouvement séparatiste Maltais, on note la
    présence active de Noreth Cal Noreth, Khan des Khans. Le fait que le chef
    des Hommes Libres affiche ouvertement ses opinions anti-impérialistes
    indique clairement que ce mouvement est loin d'être minoritaire.~[...]

    \nobreak

    [...]~Le Khan des Khans fera l'objet d'une surveillance particulière, la
    dureté de ses positions nous faisant en effet craindre pour la sécurité
    même de l'Empire.
}{Colonel Jarl Pond, Services Secrets de L'Empereur.}

\indice
{TOU-86025-HSU/514963}
{
    Le trafic de Miel est une des activités illicites les plus répandues dans
    les milieux autorisés de l'Empire, et ce malgré une répression
    particulièrement sévère.~[...]

    \nobreak

    [...]~Le nom d'Argann Terr, Intellect de l'Empereur, revient à de
    nombreuses reprises dans les rapports concernant les petits trafics de la
    Cour Impériale. Il serait utile d'interroger le suspect plus en avant afin
    de déterminer ses liens avec d'éventuels trafiquants.
}{Colonel Jarl Pond, Services Secrets de L'Empereur.}

\indice
{QTF-021489-GDR/49563}
{
    Il apparait que lors du Transfert entre Alièna~II et Méta-zerhrza (Voyage
    du 12-09-45, immatriculé AA-8361-B4-741) le groupe de trente Voyageurs
    (présents dans le Transfert en tant que simples passagers) s'est
    apparemment scindé en deux à un moment qu'il nous est impossible à
    déterminer.~[...]

    \nobreak

    [...]~Le Maitre Voyageur Eliann Quamtum et douze de ses Elèves n'ont pas
    enregistré au débarquement sur Méta-zerzha (province du Duché de Rodrigo).
    On peut donc les considérer comme disparus durant le Voyage.~[...]

    \nobreak

    [...]~La Guilde, toujours soucieuse de ses intérêts et peu prompte à
    coopérer avec nos services, a refusé d'émettre le moindre commentaire. Nous
    croyons quand même savoir que cette étrange disparition les inquiète autant
    que nous.
}{Colonel Jarl Pond, Services Secrets de L'Empereur.}

\indice
{XIG-45039-ORW/513890}
{
    \textit{Un arbre généalogique de la lignée des ducs de Romane. Un examen 
        attentif permet aisément de se rendre compte que Liadenn de Romane est
        le dernier Romane vivant, et que sa disparition entrainerait un
        changement en profondeur dans le contrôle du Duché.}
}{}

\indice
{RPJ-21806-PNS/154920}
{
    Depuis quelques années, l'Université Utty de Varley a développé un secteur
    \og entomologie\fg{} de première ordre, qui a fait de grands progrès dans
    de nombreux domaines de recherches.~[...]

    \nobreak

    [...]~Nos services ont constaté que le budget consacré à cette activité
    surpasse considérablement la normale. Interrogé à ce sujet, le Duché a
    refusé d'émettre le moindre commentaire. Nous avons placé les recherches
    des laboratoires sous surveillance, mais il semble que plusieurs d'entre
    eux, classés \og Top Secret\fg{}, échappent à nos investigations.
}{Colonel Jarl Pond, Services Secrets de L'Empereur.}

\end{document}
