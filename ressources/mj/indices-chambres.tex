\documentclass{article}
\usepackage[utf8]{inputenc}
\usepackage[T1]{fontenc}
\usepackage[francais]{babel}

\usepackage[top=2cm,bottom=2cm,left=2cm,right=2cm]{geometry}
\usepackage{framed}

\addto\captionsfrancais{\renewcommand{\figurename}{\textsc{Indice}}}
\usepackage[labelsep=endash]{caption}

\pagestyle{empty}

\newcommand{\newcharacter}{\clearpage\setcounter{figure}{0}}

\newcommand{\indice}[4]{
  \begin{figure}[H]
    \begin{center}
      \rule{0.5\textwidth}{1pt}
    \end{center}
    \begin{framed}
      \begin{samepage}
        \textit{#1}
        \nopagebreak

        #2

        \nopagebreak

        \hfill\textit{#3}
      \end{samepage}
    \end{framed}
    \caption{#4}
    \begin{center}
      \rule{0.5\textwidth}{1pt}
    \end{center}
  \end{figure}
}

\begin{document}

\indice
{(Extraits d'une lettre adressée à l'Empereur Hélios~III)}
{
    Nous, Khan des Khans, élu du peuple des Hommes Libres de Malte, nous
    adressons à l'Empereur Hélios~III, héritier du titre par droit d'ainesse,
    Grand Intendant de Malte en vertu du traité signé par Elbereth Cal
    Elbereth, premier Khan des Khans, et toujours honoré par les Hommes Libres.

    \nobreak

    [...]~Nous ne pourrons pas éternellement subir seuls les couts injustifiés
    de l'armée ducale chargée de l'exploitation du Miel.~[...]

    \nobreak

    [...]~Le sang des Hommes de Malte est précieux, et Malte a toujours payé
    son tribut. Il est inutile et dangereux de contraindre les Commandeurs des
    Ruches.~[...]

    \nobreak

    [...]~Nous, Noreth Cal Noreth, Khan des Khans, avertissons solennellement
    le Premier Représentant de l'Empire qu'il est temps de modifier les termes
    du traité, dans la paix et la discussion.~[...]
}{}
{Chambre d'Hélios~III}

\indice
{(Extraits d'une lettre accompagnant un projet de loi adressée à Hélios~III)}
{
    [...]~Comme vous nous l'aviez indiqué, le texte de loi a été rédigé en
    effaçant toutes références à l'Ordre des Pythonisses.~[...]

    \nobreak

    [...]~La loi prévoit que les nobles de l'Empire pourront se marier à des
    roturiers, sans perdre leur titres, à condition que leur conjoint
    n'appartienne pas à une association poursuivant des buts différents de ceux
    de l'Empire.~[...]

    \nobreak

    [...]~En tout état de cause, l'autorisation de l'Empereur est
    nécessaire.~[...]

    \nobreak

    [...]~Cette modification de l'adoubement pouvant amener des débats au sein
    des deux chambres, il serait bon de préparer le terrain.~[...]
}{Hubert Pendragon, Maitre des Requêtes et Lois}
{Chambre d'Hélios~III}

\indice
{(Extraits d'un mémholo)}
{
    [...]~Nous confirmons notre précédent rapport sur une activité suspecte de
    la Guilde dans le secteur \mbox{\texttt{H 234'65''4 V 984'34''23 P -
    002'57''744}}, appelé Poussières du Lion.~[...]

    \nobreak

    [...]~Aucune colonie ne se trouve dans ce secteur et aucun astéroïde n'a
    fait l'objet d'une quelconque exploitation dans ce secteur périphérique de
    l'Empire, placé sous l'autorité directe du Baron Huys, vassal du Duc de
    Martens.~[...]

    \nobreak

    [...]~Ces faits sont estimés fiables par nos experts a plus de 85\%.~[...]

    \nobreak

    [...]~Nous restons dans l'attente d'autres données que vous pourriez nous
    adresser pour nouvelle analyse.~[...]
}{Services Externes de l'Empire}
{Chambre d'Hélios~III}

\indice
{(Note adressée à l'Intendant J. Zamora)}
{
    Vous voudrez bien rédiger le décret mettant fin à la mission d'Argann Terr,
    Intellect, et prévoir les dispositions habituelles dans ces circonstances
    (cérémonie, médaille, dédommagements, pris en charge d'une résidence...).

    \nobreak

    Pas d'extra, uniquement l'usage et la coutume.
}{L'Empereur Hélios~III}
{Chambre d'Hélios~III}

\indice
{(Note adressée au Maitre des Requêtes et Lois)}
{
    Vous voudrez bien procéder à une étude exhaustive quant à la passation de
    pouvoir fixée au 50\ieme{} anniversaire de l'Héritier. S'agit-il d'une
    coutume, d'une loi? Quelles seraient les incidences éventuelles si la
    passation ne s'effectuait pas à cette date?
}{L'Empereur Hélios~III}
{Chambre d'Hélios~III}

\newcharacter

\indice
{(Extrait du dossier médical d'Hélios~III)}
{
    [...]~Nous attirons tout particulièrement votre attention sur tout
    traitement au Miel de Malte. La sensibilité de notre patient ne permet une
    absorption aux doses normales. Il vous est recommandé de prescrire des
    traitements ne contenant pas de Miel, même si ceux-ci entrainent de faibles
    désagréments pour le patient. Nous vous conseillons d'utiliser cette lettre
    en cas d'opposition du patient. Nous vous indiquons ci-après les seuils de
    tolérance au Miel du patient et le dosage qu'il convient d'utiliser. Nous
    vous \textbf{recommandons de suivre strictement la procédure quant aux
    personnes autorisées à doser le Miel}. Les conséquences d'une erreur
    seraient funestes pour le patient et, par voie de conséquence, pour les
    responsables.
}{Docteur Hari Seldon, près l'Empereur Hélios III}
{Chambre d'Andrev}

\indice
{(Extrait d'une lettre adressée au Gouverneur de l'École de Pensée Universelle
Impériale)}
{
    [...]~Nous vous serions reconnaissant de bien vouloir spécialiser votre
    élève U. Le Guin, aux fonctions d'intellect impérial attaché à ma personne.

    \nobreak

    Nous vous attribuons en conséquence un budget supplémentaire de Miel fixé
    selon le décret impérial en vigueur.

    \nobreak

    Nous exigeons un rapport strictement confidentiel sur la formation de cet
    élève.~[...]
}{Prince Andrev}
{Chambre d'Andrev}

\indice
{(Extrait de lettre accompagnant un épais rapport)}
{
    Votre Excellence, le Prince Héritier de l'Empire, cher Andrev,

    \nobreak

    [...]~Conformément à vos instructions, vous trouverez ci-joint un rapport
    économique, établi confidentiellement, sur les résultats d'une gestion
    directe de Malte, comparée à celles réalisées au cours des derniers
    siècles. Le gain économique est tout à fait significatif. Nous n'avions pas
    les moyens d'évaluer les incidences politiques d'une telle exploitation
    sans intermédiaire, mais nous supposons que vous menez une étude spécifique
    sur ce sujet.

    \nobreak

    Nous vous restons dévoués et sommes à votre disposition pour tout
    commentaire que vous souhaiteriez.
}{Votre ami, J.~Vance, au nom du groupe.}
{Chambre d'Andrev}

\newcharacter

\indice
{
    Vous trouvez un hologramme d'un petit enfant d'une dizaine d'années
    (un petit garçon). En y regardant bien, cet enfant ressemble beaucoup à
    Hélios~III.
}{}{}
{Chambre de Dame Perduinn}

\indice
{(Extrait d'une lettre manuscrite, visiblement souvent manipulée. Vous
reconnaissez le papier spécial du Palais Impérial)}
{
    Ma chérie, depuis que tu m'as annoncé la merveilleuse nouvelle, mon âme
    et mon cœur ne connaissent plus de répit. Pourquoi garder tout ce bonheur
    secret plus longtemps? Tu connais mon amour (comment pourrais-tu l'ignorer,
    ma tendre amie?). Laisse-moi tout révéler au grand jour, et accepte de
    partager ma vie, même si je sais tout ce qu'il doit t'en couter.~[...]
}{Celui qui t'aime et qui t'aimera toujours}
{Chambre de Dame Perduinn}

\indice
{(Extrait d'une lettre manuscrite. Vous reconnaissez le papier spécial de
l'ordre des Pythonisses)}
{
    [...]~Ma Fille, vous ne pouvez plus longtemps vous tromper, et pire, nous
    tromper. Je comprends les sentiments qui vous habitent et leur haute
    noblesse. Mais le chemin de la Lumière est parfois douloureux.~[...]

    \nobreak

    [...]~Puisque vous ne voulez pas revenir sur votre décision et rendre au
    Néant l'enfant que vous portez, acceptez votre retrait de notre Ordre, et
    n'entachez pas par votre conduite notre Vénérable Institution.~[...]
}{Révérende-Mère Drowen}
{Chambre de Dame Perduinn}

\indice
{(Extrait d'une lettre manuscrite. Vous reconnaissez le papier spécial de
l'ordre des Pythonisses)}
{
    [...]~Il est vrai que notre Ordre ne vit pas ses meilleurs jours et que la
    pression de l'Empire (et surtout de l'Empereur) se fait chaque jour plus
    forte. Cependant, il ne faut pas, mes sœurs, se laisser emporter par de
    sinistres pensées. L'Empire a besoin de nous autant que nous avons besoin
    de l'Empire. Sachez que j'emploierai toutes les ressources à ma disposition
    pour lutter contre les projets de l'Empereur, en particulier contre cette
    absurde \og Charte d'adoubement\fg{}, qui pourrait s'avérer une arme
    redoutable contre nous. Comptez sur mon appui dans toutes les circonstances
    à venir.~[...]
}{Révérende-Mère Perduinn}
{Chambre de Dame Perduinn}

\newcharacter

\indice
{(Texte d'un bloc mémholo de la chambre)}
{
    Nous confirmons que toutes les analyses prouvent que la vitesse atteinte
    par l'objet est supérieure de 30\% à celle du croiseur rapide Flash Gordon,
    détenteur du record de vitesse au sein de la Guilde. À ce stade d'étude,
    nous pouvons affirmer que nos données sur l'objet sont insuffisantes pour
    élaborer d'autres pistes.
}{Clifford D. Simak, Maitre des Recherches}
{Chambre de Maitre Korrig}

\indice
{(Extraits d'un recueil intitulé \textit{Intérêts et Stratégies de la Guilde})}
{
    [...]~La puissance de la Guilde, en cas de conflit majeur, est égale à
    l'ensemble des flottes ducales et impériales.~[...]

    \nobreak

    [...]~Les probabilités de défaite sont de l'ordre de 5\%, ce qui
    correspond à l'hypothèse d'une flotte interducale et impériale unie sous un
    même commandement.~[...]

    \nobreak

    [...]~Dans toutes les simulations, il est vital de s'emparer de Malte le
    plus rapidement possible, l'idéal étant de diriger Malte avant le conflit.
}{Pour la Conférence de l'Avenir, Le Rapporteur, A.~Van~Vogt}
{Chambre de Maitre Korrig}

\indice
{
    Vous trouvez dans la chambre un hologramme d'une créature non humaine. En
    faisant défiler plusieurs holos, vous acquérez la conviction qu'il s'agit
    d'un extraterrestre pilotant un vaisseau galactique d'origine non humaine.
    Aucun commentaire n'accompagne les holos.
}{}{}
{Chambre de Maitre Korrig}

\newcharacter

\indice
{(Extrait du Précis Médical à l'usage des Intellects)}
{
    [...]~Nous vous recommandons de respecter le dosage du Miel prescrit par
    l'équipe de médecins de l'École de Pensée Universelle Impériale.

    \nobreak

    [...]~Les conséquences d'une surdose sont la dépendance au Miel et la mort
    par overdose dans un délai variant de 3 à 7 ans selon les individus.
}{}
{Chambre d'Argann Terr}

\indice
{(Extraits d'un mémholo de la chambre)}
{
    [...]~Il faut peser les avantages et les inconvénients d'un contrat avec
    l'un ou l'autre des Duchés et corréler cette analyse avec leur probabilité
    respective de contrôler Malte.~[...]

    \nobreak

    [...]~Compte tenu du temps disponible, je vais devoir faire des hypothèses
    avant de me livrer à une Analyse Déductive Intellect.~[...]

    \nobreak

    [...]~Il me faut trouver la matière première seul, l'Empereur compte les
    doses.~[...]
}{}
{Chambre d'Argann Terr}

\indice
{(Extraits d'une note à l'Empereur)}
{
    [...]~Vous connaissez l'indice de corrélation prouvant que les Pythonisses
    utilisent les données et analyses des intellects de l'Empire pour effectuer
    à posteriori des fausses divinations.~[...]

    \nobreak

    [...]~L'Analyse Déductive que j'ai effectuée dernièrement prouve, sans
    l'ombre d'un doute, que cet ordre mystico-religieux professe des théories
    fumeuses sur l'avenir que doit se choisir l'Humanité. Aucune preuve
    scientifique n'est apportée sur leur prétendu Don de Divination. Il relève
    beaucoup plus du charlatanisme et de la crédulité des foules que d'une
    bonne analyse dont tout homme est capable.~[...]

    \nobreak

    [...]~Nous vous recommandons de ne pas approcher de représentante de cette
    secte afin de ne pas augmenter leur influence, et vous supplions une
    nouvelle fois de prendre des mesures draconiennes pour éradiquer ce fléau
    de l'obscurantisme dans l'Empire.
}{Argann Terr, votre dévoué intellect}
{Chambre d'Argann Terr}

\newcharacter

\indice
{
    Vous trouvez un petit holo-video. C'est un essai militaire publié à compte
    d'auteur. Son titre: \og De l'intérêt de la maitrise des vaisseaux de la
    Guilde dans un conflit armé à l'échelle Impériale\fg{}. Il est signé Deleb
    Karnar.
}{}
{}
{Chambre de Deleb Karnar}

\indice
{(Vous reconnaissez le papier spécial du Palais Impérial)}
{
    [...]~Par la présente, Nous, Empereur Hélios~III, élevons le Général Deleb
    Karnar au rang de Maréchal d'Empire. De plus, Nous lui confions le titre
    d'Instructeur Général des cadets de l'École Supérieur de Stratégie
    Spatiale.~[...]

    \nobreak

    [...]~Il devra prendre son poste le plus rapidement possible et abandonner
    ses fonctions de Maitre de Guerre auprès du Duché de Rodrigo.~[...]

    \nobreak

    \textit{(En bas du document, une note manuscrite)}

    \nobreak

    Il me prend pour qui? Inacceptable!
}{}
{Chambre de Deleb Karnar}

\indice
{(Extraits d'un épais rapport de la Commission des Finances Impériale)}
{
    [...]~Nos inspecteurs financiers, après recoupement avec l'audit mené par
    le Service de Contrôle des Voyages, sont arrivés à la même conclusion: les
    livraisons de Miel, telles qu'elles sont consignées dans les registres de
    comptabilité de l'Astroport de Malte~A-II, ne correspondent pas aux
    déclarations de productions des Khans telles qu'on peut les lire dans les
    registres des Maitres Producteurs Maltais.~[...]

    \nobreak

    [...]~Ces écarts de livraisons étant toujours en solde négatif, il nous
    apparait qu'on ne saurait écarter l'hypothèse d'une fraude délibérée sur
    les exportations du Miel.~[...]

    \nobreak

    [...]~Pourriez-vous nous fournir rapidement les mains courantes des
    patrouilles de surveillance et de contrôle de la Production Maltaise qui
    sont sous votre haute responsabilité?~[...]
}{Del Tellim, Inspecteur général aux finances de l'Empire.}
{Chambre de Deleb Karnar}

\newcharacter

\indice
{(Une lettre manuscrite)}
{
    [...]~Mon Fils, chair de Malte et chair du Duché, n'oublie jamais les
    paroles de celui qui croyait être ton père: \og un Khan, ce n'est pas
    seulement un Homme Libre. C'est celui qui représente le Clan, celui qui
    voit plus haut que tous les siens, celui qui marche plus loin, celui qui
    parle plus fort. C'est l'Honneur, la Justice, la Sagesse. Et parmi eux
    tous, le Khan des Khans doit être le meilleur\fg{}.~[...]

    \nobreak

    [...]~Toi qu'on a choisi pour nous représenter en ces temps si troublés,
    toi justement dont le sang n'est pas pur, par ma faute, mon péché, sache
    que mon amour t'accompagne et que seul le bonheur de Malte doit guider tes
    pas.~[...]
}{Celle pour qui tu es Tout, ta Mère.}
{Chambre de Noreth Cal Noreth}

\indice
{(Une lette manuscrite)}
{
    [...]~Nous, Hommes Libres de Malte, ne pourrons tolérer pendant des siècles
    le joug intolérable de ceux qui viennent des étoiles. Un jour, un jour très
    proche, nous nous lèverons et nous montrerons à l'Empire et à ses sbires le
    visage de l'Homme Libre.~[...]

    \nobreak

    [...]~Le Destin de Malte ne repose plus que sur notre détermination et
    notre volonté. Nos Anciens ne sont pas morts pour rien, le sang qui coulent
    en nous est le leur. Demain, Malte sera libre et nos fils riront sous le
    soleil de notre triomphe.~[...]
}{Noreth Cal Noreth\\
(En bas du document, une note)\\
\textbf{Je les convaincrai!}}
{Chambre de Noreth Cal Noreth}

\indice
{(Extraits d'un rapport au Conseil des Khans)}
{
    [...]~Même s'il reste difficile d'évaluer avec précision le volume de Miel
    concerné, les pertes habituelles dues au maniement de notre Or par des
    mains étrangères ne sauraient, à elles seules, expliquer les écarts
    continuels constatés ces dernières années. Toutes les vérifications faites
    par nos Frères concordent: la part de Miel manquante sur les récoltes qui
    viennent de s'écouler ne peut s'expliquer que par un détournement
    systématique et organisé.~[...]

    \nobreak

    [...]~Il est d'ailleurs proprement incroyable que les Étrangers ne se
    soient rendu compte de rien de leur côté. Est-ce une simple preuve
    supplémentaire de leur aveuglement et de leur balourdise?~[...]

    \nobreak

    [...]~L'enquête que nous menons conduira à la découverte des coupables et à
    une prise de sanctions exemplaires, et ce, quel que soit le nom ou le titre
    que porte l'auteur de ce crime inqualifiable.
}{Noreth Cal Noreth, Khan des Khans.}
{Chambre de Noreth Cal Noreth}

\indice
{(Extraits d'un rapport au Conseil des Khans)}
{
    [...]~Les Ruches signalées par le Frère Kel Merqaz n'ont pu être
    localisées, et ce malgré des recherches très poussées sur tous les
    territoires de floraison. Il semble bien que nous devions admettre que nous
    sommes devant un nouvel élément de \og l'affaire des Orchidées
    volées\fg{}.~[...]

    \nobreak

    [...]~Cette affaire est d'une telle importance et les conséquences
    pourraient être d'une telle ampleur que la plus grande discrétion est
    évidemment de circonstance. Bien sûr, les Étrangers ne doivent jamais être
    mis au courant avant que nous ayons résolu ce mystère.~[...]

    \nobreak

    [...]~Les sanctions que nous seront amenés à prendre suite à cette affaire
    feront date dans l'histoire pourtant mouvementée de Malte.
}{Noreth Cal Noreth, Khan des Khans.}
{Chambre de Noreth Cal Noreth}

\newcharacter

\indice
{(Extraits d'un audit impérial sur la comptabilité du Duché en matière de
Miel)}
{
    [...]~Notre analyse nous conduit à penser qu'une dissimulation de Miel par
    l'une ou plusieurs de vos divisions basées sur Malte a été opérée à votre
    insu.~[...]

    \nobreak

    [...]~Nous vous demandons d'apporter tous les éclaircissements possibles
    sur cette dissimulation, et de veiller au retour à la normale de
    l'acheminement de l'intégralité de la production destinée à l'Empire.
}{Philip José Farmer, Trésorier Impérial en charge du Malte}
{Chambre du Duc Igane de Rodrigo}

\indice
{(Extrait d'une note classée CSD dont le sceau est cassé)}
{
    [...]~Conformément au planning prévu, la formation des Voyageurs s'effectue
    sans aucun problème et suit son déroulement normal.

    \nobreak

    [...]~Il semblerait que nous ayons un surplus de Matière importée par
    rapport à nos prévisions de consommation.~[...]
}{Raïs El Andulus}
{Chambre du Duc Igane de Rodrigo}

\indice
{(Lettre en provenance du Duché de Rodrigo)}
{
    Cher Duc,

    \nobreak

    Les résultats de l'Analyse que vous m'avez demandée d'actualiser restent
    malheureusement identiques aux précédents:

    \nobreak

    À paramètres constants, votre excellence ne dirigera pas Malte pour la
    prochaine période. Cette Analyse est fiable à plus de 99\%.

    \nobreak

    Les évènements pouvant modifier les paramètres de l'Analyse restent donc
    les mêmes. Vous jugerez de l'opportunité en fonctions des scénario élaborés
    et simulés.
}{Fizban Fistandantilus, Intellect ducal.}
{Chambre du Duc Igane de Rodrigo}

\newcharacter

\indice
{(Hologrammme Excel Impérium à haute résolution supra objective)}
{
    Il s'agit d'un holo d'une femme âgée appartenant sans aucun doute à la
    Maison des Romane mais qui n'appartient pas à la lignée ducale. En tous
    cas, elle n'en porte pas les signes distinctifs.

    \nobreak

    La qualité de l'hologramme vous permet de supposer qu'il s'agit d'un être
    cher au Duc.
}{}
{Chambre du Duc Liadenn de Romane}

\indice
{(Extrait d'une directive adressée à Elric Stormbringer, Maitre de Guerre)}
{
    [...]~Nous vous demandons de prendre toutes les dispositions en vue de
    protéger les colonies et planètes du Duché en préparant une attaque sur les
    objectifs mentionnés lors du dernier conseil de guerre.~[...]

    \nobreak

    [...]~Vous devez être opérationnel dans les heures qui suivront mon
    commandement.~[...]

    \nobreak

    [...]~Vous disposerez, en attente près de Carbon~B, une force
    d'intervention réduite quant au nombre de vaisseaux, mais dont la puissance
    d'attaque et la célérité doivent être supérieures à la flotte impériale.
}{Duc Liadenn de Romane}
{Chambre du Duc Liadenn de Romane}

\indice
{(Extraits d'une lettre adressée à l'Empereur Hélios~III)}
{
    [...]~Nous voudrions évoquer avec vous les clauses de passation du titre
    Ducal au sein des nobles familles.~[...]

    \nobreak

    [...]~Nous trouvons trop contraignant, voire dangereux dans certains cas,
    de confier la responsabilité du Duché en fonction du droit d'ainesse. Nous
    vous adressons pour mémoire les cas de conflits qui se sont produits ces
    derniers siècles.~[...]

    \nobreak

    [...]~Nous vous proposons qu'une nouvelle loi soit ratifiée par le Sénat
    Impérial et par la Haute Cour des Nobles afin que le conseil de famille
    désigne le successeur des Ducs parmi la famille ducale. Vous trouverez en
    annexe une proposition d'article de loi.~[...]

    \nobreak

    [...]~Nous comptons sur votre soutien pour moderniser l'un des derniers
    archaïsme de l'Empire.
}{Votre dévoué vassal, le Duc Liadenn de Romane}
{Chambre du Duc Liadenn de Romane}

\newcharacter

\indice
{(Extraits d'un rapport financier)}
{
    [...]~Le ratio endettement/produit intérieur a atteint, depuis trois ans,
    un niveau tel qu'il parait difficile pour le Duché de pouvoir espérer un
    retour à la normale sans un concours de circonstances exceptionnel.~[...]

    \nobreak

    [...]~Dans les conditions économiques actuelles, une intervention Impériale
    dans les affaires du Duché apparait comme inéluctable, avec toutes les
    conséquences fâcheuses pour la Maison Ducale que vous pouvez
    imaginer.~[...]
}{Arben Toht, Grand Conseiller aux comptes de la Maison Varley.}
{Chambre du Duc Mika de Varley}

\indice
{(Extrait d'un rapport technique)}
{
    [...]~Les derniers tests effectués sur les prélèvements du groupe
    \og clone\fg{} révèlent un net progrès quantitatif. Les corrélations entre
    \og origine\fg{} et \og clone\fg{} sur le test de chromatographie nucléaire
    Panchkin/Mercloss (appliqué sur la partie florale) démontrent une
    compatibilité de 98,31\%. Ces résultats nous permettent de penser que la
    réussite du projet n'est plus maintenant qu'une question de temps et de
    moyens.~[...]
}{(Pas de signature)}
{Chambre du Duc Mika de Varley}

\indice
{(Extrait d'un rapport technique)}
{
    [...]~Un net progrès se dessine depuis la dernière intervention de l'équipe
    Recherche et Développement. En particulier, la réussite de la fécondation
    de la reine et la viabilité des larves nous permettent désormais
    d'envisager sérieusement la prochaine étape: l'autonomie complète d'une
    ruche.~[...]
}{(Pas de signature)}
{Chambre du Duc Mika de Varley}

\end{document}
