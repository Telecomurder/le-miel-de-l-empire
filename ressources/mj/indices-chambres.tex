\documentclass{article}
\usepackage[utf8]{inputenc}
\usepackage[T1]{fontenc}
\usepackage[francais]{babel}

\usepackage[top=2cm,bottom=2cm,left=2cm,right=2cm]{geometry}
\usepackage{framed}

\addto\captionsfrancais{\renewcommand{\figurename}{\textsc{Indice}}}
\usepackage[labelsep=endash]{caption}

\pagestyle{empty}

\newcommand{\newcharacter}{\clearpage\setcounter{figure}{0}}


\newcommand{\indice}[4]{
    \begin{figure}[H]
        \begin{center}
            \rule{0.5\textwidth}{1pt}
        \end{center}
        \begin{framed}
            \begin{samepage}
                \textit{#1}
                \nopagebreak

                #2

                \nopagebreak

                \hfill\textit{#3}
            \end{samepage}
        \end{framed}
        \caption{#4}
        \begin{center}
            \rule{0.5\textwidth}{1pt}
        \end{center}
    \end{figure}
}

\begin{document}

\indice
{(Extraits d'une lettre adressée à L'Empereur Hélios III)}
{Nous, Khân des Khâns, élu du peuple des Hommes Libres de Malte, nous adressons
à l'Empereur Hélios~III, héritier du titre par droit d'aînesse, Grand Intendant
de Malte en vertu du traité signé par Elbereth Cal Elbereth, premier Khân des
Khâns, et toujours honoré par les Hommes Libres.

\nobreak

[...]~Nous ne pourrons pas éternellement subir seuls les coûts injustifiés de
l'armée ducale chargée de l'exploitation du Miel.~[...]

\nobreak

[...]~Le sang des Hommes de Malte est précieux, et Malte a toujours payé son
tribut. Il est inutile et dangereux de contraindre les Commandeurs des
Ruches.~[...]

\nobreak

[...]~Nous, Noreth Cal Noreth, Khân des Khâns, avertissons solennellement le
Premier Représentant de l'Empire qu'il est temps de modifier les termes du
traité, dans la paix et la discussion.~[...]}
{}
{Chambre d'Hélios~III}

\indice
{(Extraits d'une lettre accompagnant un projet de loi adressée à Hélios III)}
{[...]~Comme vous nous l'aviez indiqué, le texte de loi a été rédigé en
    effaçant toutes références à l'Ordre des Pythonisses.~[...]

\nobreak

[...]~La loi prévoit que les nobles de l'Empire pourront se marier à des
roturiers, sans perdre leur titres, à condition que leur conjoint
n'appartienne pas à une association poursuivant des buts différents de ceux de
l'Empire.~[...]

\nobreak

[...]~En tout état de cause, l'autorisation de l'Empereur est nécessaire.~[...]

\nobreak

[...]~Cette modification de l'adoubement pouvant amener des débats au sein des
deux chambres, il serait bon de préparer le terrain.~[...]}
{Hubert Pendragon, Maître des Requêtes et Lois}
{Chambre d'Hélios~III}

\indice
{(Extraits d'un mémholo)}
{[...]~Nous confirmons notre précédent rapport sur une activité suspecte de la
Guilde dans le secteur \mbox{\texttt{H 234'65''4 V 984'34''23 P -
002'57''744}}, appelé Poussières du Lion.~[...]

\nobreak

[...]~Aucune colonie ne se trouve dans ce secteur et aucun astéroïde n'a fait
l'objet d'une quelconque exploitation dans ce secteur périphérique de l'Empire,
placé sous l'autorité directe du Baron Huys, vassal du Duc de Martens.~[...]

\nobreak

[...]~Ces faits sont estimés fiables par nos experts a plus de 85\%.~[...]

\nobreak

[...]~Nous restons dans l'attente d'autres données que vous pourriez nous
adresser pour nouvelle analyse.~[...]} 
{Services Externes de l'Empire}
{Chambre d'Hélios~III}

\indice
{(Note adressée à L'Intendant J. Zamora)}
{Vous voudrez bien rédiger le décret mettant fin à la mission d'Argann Terr,
Intellect, et prévoir les dispositions habituelles dans ces circonstances
(cérémonie, médaille, dédommagements, pris en charge d'une résidence...).

\nobreak

Pas d'extra, uniquement l'usage et la coutume.}
{L'Empereur Hélios~III}
{Chambre d'Hélios~III}

\indice
{(Note adressée au Maître des Requêtes et Lois)}
{Vous voudrez bien procéder à une étude exhaustive quant à la passation de
pouvoir fixée au 50\ieme{} anniversaire de l'Héritier. S'agit-il d'une coutume,
d'une loi ? Quelles seraient les incidences éventuelles si la passation
ne s'effectuait pas à cette date ?}
{L'Empereur Hélios~III}
{Chambre d'Hélios~III}

\newcharacter

\indice
{(Extrait du dossier médical d'Hélios~III)}
{[...]~Nous attirons tout particulièrement votre attention sur tout traitement
au Miel de Malte. La sensibilité de notre patient ne permet une absorption aux
doses normales. Il vous est recommandé de prescrire des traitements ne
contenant pas de Miel, même si ceux-ci entraînent de faibles désagréments pour
le patient. Nous vous conseillons d'utiliser cette lettre en cas d'opposition
du patient. Nous vous indiquons ci-après les seuils de tolérance au Miel du
patient et le dosage qu'il convient d'utiliser. Nous vous \textbf{recommandons
de suivre strictement la procédure quant aux personnes autorisées à doser le
Miel}. Les conséquences d'une erreur seraient funestes pour le patient et, par
voie de conséquence, pour les responsables.}
{Docteur Hari Seldon, près l'Empereur Hélios III}
{Chambre d'Andrev}

\indice
{(Extrait d'une lettre adressée au Gouverneur de l'École de Pensée Universelle
Impériale)}
{[...]~Nous vous serions reconnaissant de bien vouloir spécialisé votre élève
U. Le Guin, aux fonctions d'intellect impérial attaché à ma personne.

\nobreak

Nous vous attribuons en conséquence un budget supplémentaire de Miel fixé selon
le décret impérial en vigueur.

\nobreak

Nous exigeons un rapport strictement confidentiel sur la formation de cet
élève.~[...]}
{Prince Andrev}
{Chambre d'Andrev}

\indice
{(Extrait de lettre accompagnant un épais rapport)}
{Votre Excellence, le Prince Héritier de l'Empire, cher Andrev,

\nobreak

[...]~Conformément à vos instructions, vous trouverez ci-joint un rapport
économique, établi confidentiellement, sur les résultats d'une gestion directe
de Malte, comparée à celles réalisées au cours des derniers siècles. Le gain
économique est tout à fait significatif. Nous n'avions pas les moyens d'évaluer
les incidences politiques d'une telle exploitation sans intermédiaire, mais
nous supposons que vous menez une étude spécifique sur ce sujet.

\nobreak

Nous vous restons dévoués et sommes à votre disposition pour tout commentaire
que vous souhaiteriez.}
{Votre ami, J.~Vance, au nom du groupe.}
{Chambre d'Andrev}

\newcharacter

\indice
{Vous trouvez un hologramme d’un petit enfant d’une dizaine d’années (un petit
garçon). En y regardant bien, cet enfant ressemble beaucoup à Hélios~III.}
{}
{}
{Chambre de Dame Perduinn}

\indice
{(Extrait d’une lettre manuscrite, visiblement souvent manipulée. Vous
reconnaissez le papier spécial du Palais Impérial)}
{Ma chérie, depuis que tu m’as annoncé la merveilleuse nouvelle, mon âme
et mon cœur ne connaissent plus de répit. Pourquoi garder tout ce bonheur
secret plus longtemps ? Tu connais mon amour (comment pourrais-tu l’ignorer,
ma tendre amie ?). Laisse-moi tout révéler au grand jour, et accepte de
partager ma vie, même si je sais tout ce qu’il doit t’en coûter.~[...]}
{Celui qui t’aime et qui t’aimera toujours}
{Chambre de Dame Perduinn}

\indice
{(Extrait d’une lettre manuscrite. Vous reconnaissez le papier spécial de
l’ordre des Pythonisses)}
{[...]~Ma Fille, vous ne pouvez plus longtemps vous tromper, et pire, nous
tromper. Je comprends les sentiments qui vous habitent et leur haute noblesse.
Mais le chemin de la Lumière est parfois douloureux.~[...]

\nobreak

[...]~Puisque vous ne voulez pas revenir sur votre décision et rendre au Néant
l’enfant que vous portez, acceptez votre retrait de notre Ordre, et n’entachez
pas par votre conduite notre Vénérable Institution.~[...]}
{Révérende-Mère Drowen}
{Chambre de Dame Perduinn}

\indice
{(Extrait d’une lettre manuscrite. Vous reconnaissez le papier spécial de
l’ordre des Pythonisses)}
{[...]~Il est vrai que notre Ordre ne vit pas ses meilleurs jours et que la
pression de l’Empire (et surtout de l’Empereur) se fait chaque jour plus forte.
Cependant, il ne faut pas, mes s\oe{}urs, se laisser emporter par de sinistres
pensées. L’Empire a besoin de nous autant que nous avons besoin de l’Empire.
Sachez que j’emploierai toutes les ressources à ma disposition pour lutter
contre les projets de l’Empereur, en particulier contre cette absurde <<~Charte
d’adoubement~>>, qui pourrait s’avérer une arme redoutable contre nous. Comptez
sur mon appui dans toutes les circonstances à venir.~[...]}
{Révérende-Mère Perduinn}
{Chambre de Dame Perduinn}

\newcharacter

\indice
{(Texte d'un bloc mémholo de la chambre)}
{Nous confirmons que toutes les analyses prouvent que la vitesse atteinte par
l'objet est supérieure de 30\% à celle du croiseur rapide Flash Gordon,
détenteur du record de vitesse au sein de la Guilde. À ce stade d'étude, nous
pouvons affirmer que nos données sur l'objet sont insuffisantes pour élaborer
d'autres pistes.}
{Clifford D. Simak, Maitre des Recherches}
{Chambre de Maitre Korrig}

\indice
{(Extraits d'un recueil intitulé <<~Intérêts et Stratégies de la Guilde~>>)}
{[...]~La puissance de la Guilde, en cas de conflit majeur, est égale à
l'ensemble des flottes ducales et impériales.~[...]

\nobreak

[...]~Les probabilités de défaite sont de l'ordre de 5\%, ce qui correspond à
l'hypothèse d'une flotte interducale et impériale unie sous un même
commandement.~[...]

\nobreak

[...]~Dans toutes les simulations, il est vital de s'emparer de Malte le plus
rapidement possible, l'idéal étant de diriger Malte avant le conflit.}
{Pour la Conférence de l'Avenir, Le Rapporteur, A.~Van~Vogt}
{Chambre de Maitre Korrig}

\indice
{Vous trouvez dans la chambre un hologramme d'une créature non humaine. En
faisant défiler plusieurs holos, vous acquérez la conviction qu'il s'agit d'un
extraterrestre pilotant un vaisseau galactique d'origine non humaine. Aucun
commentaire n'accompagne les holos.}
{}
{}
{Chambre de Maitre Korrig}

\end{document}

