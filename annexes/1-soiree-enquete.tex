\documentclass[14pt,twocolumn]{extarticle}
\usepackage[utf8]{inputenc}
\usepackage[T1]{fontenc}
\usepackage[francais]{babel}

\setlength{\columnsep}{0.5in}

\usepackage{geometry}
\geometry{a4paper}
\geometry{margin=1in}

\usepackage{times}

\usepackage{titlesec}
\titleformat{\section}[frame]
{\normalfont} {} {15pt} {\itshape\bfseries\filcenter}
\titleformat{\subsection}
{\normalfont} {} {1pt} {\itshape\bfseries}

\pagestyle{empty}

\begin{document}

\section{UNE SOIRÉE ENQUÊTE~: QU'EST-CE DONC~?}

Une aventure, un mystère, une soirée dont vous allez être l'un des personnages
principaux~! Ce personnage va être plongé au c\oe{}ur d'une intrigue qui promet
d'être palpitante~: un vol, un meurtre ou autre. Vous devrez, sans doute avec
l'aide des autres invités (et bien sûr de l'organisateur), dénouer les fils de
l'intrigue et démasquer le coupable qui se trouve forcément parmi les invités.

\subsection{VOTRE PERSONNAGE}

Votre personnage est décrit dans le document qui est joint à ce texte. Nous
avons résumé les traits de sa personnalité ainsi que ses connaissances et ses
rapports avec les autres personnages. Lisez attentivement ce texte qui devra
vous servir de source d'inspiration pour mieux <<~sentir~>> votre personnage.
Bien sûr, si vous trouvez que le rôle ne vous convient guère, vous êtes libre
de l'adapter à votre propre personnalité, en respectant toutefois l'intrigue de
la soirée (n'hésitez pas à en parler aux organisateurs, ils sont là pour ça).

Votre personnage poursuit plusieurs buts~: le principal est bien évidemment de
résoudre l'énigme de la soirée et de trouver le coupable. Mais il a aussi ses
objectifs spécifiques que vous ne devrez pas perdre de vue. À vous de trouver
l'équilibre entre ces deux activités. Pas toujours facile~!

Il est important que tout au long de la soirée vous respectiez une règle
essentielle~: vous n'êtes pas vous mais le personnage que vous jouez. Par
exemple ne tutoyez pas vos amis ou votre conjoint, ne parlez pas de votre
dernier week-end en Normandie, etc. Vous devez agir comme votre personnage
agirait. Si tout le monde respecte cette règle de base, vous verrez que la
soirée sera beaucoup plus amusante pour tout le monde~!

Vous allez vite vous rendre compte qu'il est quasiment impossible de résoudre
l'enquête en solitaire. Il va falloir discuter, échanger des informations avec
les autres joueurs (peut-être avec le coupable). Sachez rester prudent, mais on
n'a rien sans rien~: plus vous parlerez, plus vous en saurez aussi. Et vous
verrez que c'est au fil des discussions que l'enquête va peu à peu avancer.

\subsection{LE DÉROULEMENT DE LA SOIRÉE}

Dès le début de la soirée, vous allez jouer votre personnage et vous serez
plongé au c\oe{}ur de l'intrigue. Pour vous mettre dans l'ambiance, vous
trouverez ci-joint, en plus du texte vous décrivant, un synopsis décrivant le
monde dans lequel vous allez évoluer, la liste des autres invités et leurs
descriptions, un certain nombre de Points d'Action, et si nécessaire des
documents que vous êtes le seul à détenir. Lisez attentivement ces documents
qui vont vous aidez à vous y retrouver. Si certains points vous paraissent
obscurs n'hésitez pas à contacter les organisateurs.

\subsection{COMMENT MENER L'ENQUÊTE}

Votre principale occupation dans la soirée sera probablement le bavardage avec
les autres joueurs. C'est en discutant que vous allez en apprendre plus sur eux
et faire progresser l'enquête.

Une autre source d'information importante se trouve dans des indices cachés
dans les pièces virtuelles du lieu où se déroule la soirée.

\subsection{LES POINTS D'ACTIONS}

Vous avez reçu un certain nombre de Points d'Action. Vous pouvez utiliser ces
PA pour accomplir les actions qui sont décrites dans votre feuille de
personnage. Certains joueurs ont des <<~talents~>> un peu spéciaux qui leur
sont expliqués. Une action commune à tous est la <<~fouille~>>.

\subsection{FOUILLER UNE PIÈCE}

Le lieu où se déroule la soirée est évidemment assez petit~: un appartement ou
une maison. Dans le scénario le lieu de la soirée est beaucoup plus grand. La
pièce où vous allez jouer est la pièce principale, les autres sont des pièces
virtuelles. Vous pouvez les fouiller car des indices s'y trouvent certainement
cachés. La procédure est simple~: vous prenez un organisateur à part en lui
indiquant sur le plan la pièce que vous voulez fouiller. En échange du ou des
PA nécessaires, l'organisateur vous remettra un indice et un seul. Ce peut être
n'importe quoi~: un livre, une statue, une odeur... Attention, vous n'êtes pas
le seul à fureter et les autres joueurs sont peut-être déjà passé dans cette
pièce (rassurez-vous, il y a certainement plus d'un indice par pièce)~!

Tous les joueurs vont peut-être vouloir <<~fouiller~>> en même temps (vous êtes
mesquins parfois). Merci d'être fair-play et d'attendre votre tour pour ne pas
rendre fous les organisateurs~!

Les Points d'Action peuvent aussi servir de monnaie d'échange (ou de pression).
Après tout pourquoi fouiller si pour le même nombre de PA un autre joueur vous
montre un indice~? <<~L'argent~>> est peut-être important pour certains
d'entre vous, ne l'oubliez pas. À vous de savoir dépenser vos PA pour le mieux.

\textbf{Important}~: pour des raisons de confort, les organisateurs ont
définitivement écarté l'idée de dissimuler pour de vrai des indices réels dans
la pièce de jeu. Inutile donc de retourner les coussins du canapé ou de
fouiller derrière la télé~: il n'y a rien à trouver dans la pièce de jeu. Merci
d'avance pour le maitre de maison.

Vous avez certainement constaté que vous êtes déjà en possession de quelques
documents. Une situation qui va s'aggraver au fur et à mesure que vous
trouverez des indices. Il serait bon que vous ameniez à la soirée de quoi
ranger tout ça (un sac à main, une pochette, un porte-documents...). De quoi
écrire sera surement utile aussi.

Bien évidemment, le fair-play est de mise~: si vous tombez par hasard sur une
feuille qui ne vous appartient pas (tombée d'un porte-documents par exemple, ne
souriez pas, ça arrive), il est de bon ton de ne pas la lire et de la remettre
à un organisateur. De même l'espionnage d'une conversation privée est du plus
mauvais gout. Soyez <<~classe~>>~!

\subsection{LA FIN}

Un évènement ou une information déterminera la fin de la soirée. Vous pourrez
alors redevenir vous-même. Les organisateurs vous poseront alors quelques
questions qui serviront à déterminer le détective le plus perspicace. Selon le
résultat le coupable sera considéré comme démasqué ou au contraire comme
échappant en ricanant aux foudres de la justice~!

Pour conclure, les auteurs tiennent à remercier les éditions \textit{Sans Peur
et Sans Reproche} (SPSR) qui, à défaut d'être les inventeurs de la
murder-party, ont tout de même été la source d'inspiration pour la façon
d'organiser ce scénario. Nous encourageons les joueurs qui vont aimé cette
soirée à se procurer les scénarios SPSR (disponibles en libre accès sur
Internet) et à les jouer.

\end{document}
