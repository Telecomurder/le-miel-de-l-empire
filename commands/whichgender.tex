\usepackage{ifthen}

% Une commande pour réaliser un test ternaire sur le sexe du personnage.
% La commande prend quatre argument :
% * le premier est la variable de sexe à tester
% * le deuxième est le résultat de la commande si le personnage est masculin
% * le troisième est le résultat de la commande si le personnage est féminin 
% * le quatrième est le résultat de la commande si le personnage est neutre
%
% Exemple :
% \newcommand{\e}{\whichgender{\charactergender}{}{e}{(e)}}
% La commande \e ne fera rien si le personnage est masculin, ajoutera un "e"
% si le personnag est féminin, et ajoutera un "(e)" si le personnage est neutre
\newcommand{\whichgender}[4]{
    \ifthenelse{\equal{#1}{MALE}
        {#2}
        {
            \ifthenelse{\equal{#1}{FEMALE}}
            {#3}{#4}
        }
    }
}

